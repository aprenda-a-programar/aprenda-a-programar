No capítulo 1 foi referida a existência de diversos tipos de dados, que podem variar de linguagem para linguagem e também a existência de variáveis e constantes.

\begin{defi}
Os \textbf{tipos de dados} constituem uma variedade de valores e de operações que uma variável pode suportar. São necessários para indicar ao compilador (ou interpretador) as conversões necessárias para obter os dados da memória.
\end{defi}

Os tipos de dados subdividem-se ainda em dois grupos: os tipos primitivos e os tipos compostos.

\begin{defi}
\textbf{Tipos primitivos}, nativos ou básicos são aqueles que são fornecidos por uma linguagem de programação como um bloco de construção básico.
\end{defi}

\begin{defi}
\textbf{Tipos compostos} são aqueles que podem ser construídos numa linguagem de programação através de tipos de dados primitivos e compostos. A este processo denomina-se \textbf{composição}.
\end{defi}

\section{Variáveis}

No capítulo 1 foi abordada a existência de variáveis e constantes que permitem armazenar dados. Nesta secção é explicado como se devem declarar as variáveis na linguagem de programação que será utilizada ao longo do resto do livro, a linguagem C.

Relembro que variáveis permitem o armazenamento de valores que podem ser alterados durante a execução de um programa. A declaração de variáveis em C é bastante simples:

\begin{lstlisting}
tipo_de_dados nome_da_variavel;
\end{lstlisting}

Onde:

\begin{itemize}
\item \texttt{tipo\_de\_dados} corresponde ao tipo de dados que a variável vai armazenar;
\item \texttt{nome\_da\_variavel} corresponde ao nome que a variável vai tomar, à sua identificação.
\end{itemize}

Imagine que quer criar uma variável chamada \texttt{idade} do tipo inteiro. Bastaria proceder da seguinte forma:

\begin{lstlisting}
int idade;
\end{lstlisting}

\section{Constantes}

Em C, tal como noutras linguagens de programação, existem as constantes. Relembro que constantes permitem armazenar valores imutáveis durante a execução de um programa.

Existem diversas formas de declarar constantes em C. Iremos abordar as duas mais utilizadas: as constantes declaradas e as constantes definidas.

\subsection{Constantes Definidas com \texttt{\#define}}

Chamam-se Constantes Definidas àquelas que são declaradas no cabeçalho de um ficheiro. Estas são interpretadas pelo pré-processador que procederá à substituição da constante em todo o código pelo respetivo valor. A principal vantagem deste tipo de constantes é que são sempre globais. Ora veja como se define:

\begin{lstlisting}
#define identificador valor
\end{lstlisting}

Onde:

\begin{itemize}
\item \texttt{identificador} corresponde ao nome da constante que, convencionalmente, é escrito em maiúsculas e com \textit{underscore} (\_) a separar palavras;
\item \texttt{valor} corresponde ao valor que a constante armazena.
\end{itemize}

Imagine que, por exemplo, precisa de uma constante que armazene o valor do $\pi$ e que depois se calcule o perímetro de um círculo. Poderia proceder da seguinte forma:

\begin{lstlisting}
#include <stdio.h>
#define PI 3.14159
 
int main (){
 
  double r = 5.0;              
  double circle;
 
  circle = 2 * PI * r;      
  printf("%f\n", circle);
  return 0;
}
\end{lstlisting}

O que acontece quando é definida uma constante através da diretiva \texttt{\#define} é que quando o pré-compilador lê a definição da constante, substitui todas as ocorrências no código daquela constante pelo seu valor, literalmente. 

\begin{mdframed}[backgroundcolor=cinzaclaro, linewidth=0pt]
Pode utilizar a biblioteca \texttt{math.h} que tem a constante \texttt{M\_PI} com o valor do pi.
\end{mdframed}

\subsection{Constantes Declaradas com \texttt{const}}

As Constantes Declaradas, ao contrário das Constantes Definidas são, tal como o próprio nome indica, declaradas no código, em linguagem C. A declaração destas constantes é extremamente semelhante à declaração das variáveis. Apenas temos que escrever \texttt{const} antes do tipo de dados. Ora veja como se declara uma constante:

\begin{lstlisting}
const tipo nome = valor;
\end{lstlisting}

Onde:

\begin{itemize}
\item \texttt{tipo} corresponde o tipo de dados que a constante vai conter;
\item \texttt{nome} corresponde ao nome da constante;
\item \texttt{valor} corresponde ao conteúdo da constante.
\end{itemize}

Se tentar alterar o valor de uma constante durante a execução de um programa irá obter um erro. Analise então o seguinte excerto de código:

\begin{lstlisting}
#include <stdio.h>
#include <math.h>
 
int main() {
    const double goldenRatio = (1 + sqrt(5)) / 2;
 
    goldenRatio = 9; // erro. A constante não pode ser alterada.
 
    double zero = (goldenRatio * goldenRatio) - goldenRatio - 1;
    printf("%f", zero);
    return 0;
}    
\end{lstlisting}

Existem vantagens ao utilizar cada uma destas formas de declarar constantes. As constantes declaradas podem ser locais ou globais porém as definidas são sempre globais.

\begin{defi}
Uma constante/variável \textbf{local} é uma constante/variável que está restrita a uma determinada função e só pode ser utilizada na função em que é declarada.
\end{defi}

\section{Números inteiros - \texttt{int}}

Comecemos por abordar o tipo \texttt{int}. Esta abreviatura quer dizer \textit{integer number}, ou seja, número inteiro (exemplos: 4500, 5, -250). Em C, pode-se definir o intervalo em que se situam os números de cada variável do tipo \texttt{int}.

\begin{defi}
\textbf{Inicialização de variáveis} consiste em dar o primeiro valor a uma variável no código. Uma variável não tem que ser obrigatoriamente inicializada no código. Pode-se, por exemplo, declarar uma variável e dar-lhe o valor de uma leitura em que o utilizador introduz os dados, sendo a variável inicializada com os dados inseridos pelo utilizador.
\end{defi}

No seguinte exemplo pode visualizar como se declara uma variável, ou seja, como se reserva um endereço da memória RAM, do tipo \texttt{int} com o nome \texttt{a}. Seguidamente é inicializada com o valor 20 e imprimida recorrendo à função \texttt{printf}.

\begin{lstlisting}
#include <stdio.h>      

int main() {      
       
    int a;
    a = 20;      
           
    printf("Guardei o número %d.", a);  
    return 0;      
}      
\end{lstlisting}

Antes de continuar, deve ter reparado que foi utilizado um \texttt{\%d} dentro do primeiro parâmetro. Este  é substituído pelo valor da variável \texttt{a} quando é impresso no ecrã.

\subsection{Função \texttt{printf} e números inteiros}

Utiliza-se \texttt{\%d} quando se quer imprimir o valor de uma variável dentro de uma frase. A variável deve ser colocada nos parâmetros seguintes, por ordem de ocorrência. Por exemplo:

\begin{lstlisting}
#include <stdio.h>
 
int main() {
    int a, b;
    a = 20;
    b = 100;
 	
    /* imprime: O primeiro número é: 20 */
    printf("O primeiro número é: %d.\n", a); 
    /* imprime: O segundo número é: 100 */
    printf("O segundo número é: %d.\n", b); 
    /* imprime: O primeiro e segundo números são 20 e 100 */
    printf("O primeiro e segundo números são %d e %d.\n", a, b); 
  
    return 0;
}
\end{lstlisting}

\subsection{Modificadores \texttt{short} e \texttt{long}}

Este tipo de variáveis ocupa, normalmente, entre 2 a 4 \textit{bytes} na memória de um computador. E se quiser utilizar uma variável para um número pequeno? Não poderei gastar menos recursos? E se acontecer o contrário e precisar de um número maior?

\begin{mdframed}[backgroundcolor=cinzaclaro, linewidth=0pt]

\textit{Bit (Binary Digit)} é a menor unidade de informação que pode ser armazenada ou transmitida. Um \textit{bit} só pode assumir dois valores: 0 e 1, ou seja, o código binário. Os \textit{bits} são representados por \quotes{B} minúsculo.

\textit{Byte (Binary Term)} é um tipo de dados de computação. Normalmente é utilizado para especificar a quantidade de memória ou capacidade de armazenamento de algo. É representado por um \quotes{B} maiúsculo. Geralmente:

\begin{itemize}
\item \textbf{1 Byte} = 8 bits (Octeto)
\item \textbf{1/2 Byte} = 4 bits (Semiocteto)
\end{itemize}

Existem os \textbf{Prefixos Binários (IEC)} que são nomes/símbolos utilizados para medidas indicando a multiplicação da unidade, neste caso \textit{byte}, por potências de base dois. Como por exemplo os \textit{megabibytes} (MiB) que equivale a dois elevado a vinte \textit{bytes}.

Por outro lado, existem os prefixos do \textbf{Sistema Internacional de Unidades (SI)}, que são os mais comuns, que correspondem na unidade multiplicada por potências de base dez. Como por exemplo os \textit{megabytes} (MB) que correspondem a dez elevado a seis.
\end{mdframed}

Nestas situações, pode-se utilizar modificadores.

\begin{defi}
Um \textbf{modificador} consiste numa palavra-chave, numa \textit{keyword}, que se coloca antes de um elemento de forma a modificar uma propriedade do mesmo.
\end{defi}

Para alterar a capacidade de armazenamento, ou seja, o número de \textit{bytes} ocupado por uma variável do tipo \texttt{int}, podem-se utilizar os modificadores \texttt{short} e \texttt{long}. Estes permitem-nos criar variáveis que ocupem um maior ou menor número de bytes, respetivamente.

Uma variável do tipo \texttt{int}, ao assumir um número de \textit{bytes} diferentes, também está a alterar a sua capacidade de armazenamento. Assim, temos os seguintes valores:

\begin{itemize}
\item 1 \textit{byte} armazena de -128 a +127
\item 2 \textit{bytes} armazenam de -32 768 a +32 767
\item 4 \textit{bytes} armazenam de -2 147 483 648 a +2 147 483 647
\item 8 \textit{bytes} armazenam de -9 223 372 036 854 775 808 a +9 223 372 036 854 775 807
\end{itemize}

A utilização destes modificadores é feita da seguinte forma:

\begin{lstlisting}
short int nomeDaVariavel = 20; // ou "long"
\end{lstlisting}

\paragraph{Função \textit{sizeof}}

O número de \textit{bytes} atribuído utilizando um destes modificadores pode depender do computador onde o código está a ser executado. Este \quotes{problema} depende da linguagem. Existem linguagens de programação cuja capacidade de cada tipo de variável é igual em todas as máquinas.

Para descobrirmos qual o tamanho de bytes que utiliza o seu sistema, basta recorrer à função \texttt{sizeof} da seguinte forma:

\begin{lstlisting}
#include <stdio.h>
 
int main() {    
    printf("int : %d bytes\n", sizeof(int) );
    printf("short int: %d bytes\n", sizeof(short) );
    printf("long int: %d bytes\n", sizeof(long) );
    return 0;
}
\end{lstlisting}

No computador que estou a utilizar, por exemplo, \texttt{short} refere-se a 2 \textit{bytes}, \texttt{long} a 8 \textit{bytes} e o tamanho padrão de \texttt{int} é 4 \textit{bytes}.

\subsection{Modificadores \texttt{signed} e \texttt{unsigned}}

Como sabe, os números inteiros podem assumir forma positiva e negativa. Por vezes, na programação, os números negativos podem atrapalhar (ou então ajudar), dependendo do caso.

Para termos controlo sobre a \quotes{positividade} ou \quotes{negatividade} de um número, podemos atribuir os modificadores \texttt{signed} e \texttt{unsigned}. Para que uma variável possa conter tanto números positivos como negativos, devemos utilizar o modificador \texttt{signed}. Caso queira que o número seja apenas positivo, incluindo 0, utilize \texttt{unsigned}.

Tendo em conta que variáveis marcadas com o modificador \texttt{unsigned} não podem conter números negativos, podem conter um intervalo de números positivos superior ao regular. Imaginando que uma variável \texttt{int} suporta números entre -32 768 e 32 767; a mesma variável com o modificador \texttt{unsigned} irá suportar números entre 0 e 65 535.

\section{Números reais - \texttt{float} e \texttt{double}}

Além dos números inteiros, existem outros tipos de dados que nos permitem armazenar números que, ao invés de serem inteiros, são decimais (por exemplo 1,3; 5,540; etc).

Existem dois tipos de dados que nos permitem armazenar valores do tipo decimal/real, ou seja, que têm casas decimais. Estes tipos são \texttt{float} e \texttt{double}. Devem ser utilizados da seguinte forma:

\begin{lstlisting}
float pi = 3.14;
double pi = 3.14159265359;
\end{lstlisting}

Como pode ter reparado, em C (e na maioria das linguagens de programação), não se utiliza a vírgula, mas sim um ponto para separar a parte inteira da decimal.

A diferença entre \texttt{float} e \texttt{double} é que o segundo ocupa mais memória que o primeiro, logo consegue armazenar números de maior dimensão.

Normalmente, o tipo float ocupa 4 \textit{bytes} de memória RAM enquanto o segundo tipo, \texttt{double}, ocupa 8 \textit{bytes} de memória. Mais uma vez, relembro que estes valores podem alterar de máquina para máquina.

Para quem precise de fazer cálculos mais precisos, o tipo de dados \texttt{double} é o mais aconselhado pois é o que permite uma maior extensão do valor.

\subsection{Função \texttt{printf} e números reais}

Recorrendo à função \texttt{printf} abordada na secção \textbf{1.5.7}, utiliza-se \texttt{\%f} para se imprimir o valor de uma variável dos tipos \texttt{float} e \texttt{double}. A variável deve ser colocada nos parâmetros seguintes, por ordem de ocorrência. Por exemplo:

\begin{lstlisting}
#include <stdio.h>
 
int main() {    
    float piMinor = 3.14;
    double piMajor = 3.14159265359;
 	
	// imprime: Pi pode ser 3.140000 mas, de forma mais exata, é 3.141593 
    printf("Pi pode ser %f mas, de forma mais exata, é %f.", piMinor, piMajor);
    
    return 0;
}
\end{lstlisting}

Pode visualizar que, quando se utiliza \texttt{\%f}, é utilizado um número específico de casas decimais. Caso o número de casas decimais seja mais pequeno do que o da variável original, o número é arredondado.

Pode definir o número de casas decimais que quer que sejam apresentadas da seguinte forma: \texttt{\textbf{\%.\{númeroDeCasasDecimais\}f}}. Veja o seguinte exemplo, baseado no anterior:

\begin{lstlisting}
#include <stdio.h>
 
int main() {    
    float piMinor = 3.14;
    double piMajor = 3.14159265359;
 	
	// imprime: Pi pode ser 3.14 mas, de forma mais exata, é 3.14159265359.
    printf("Pi pode ser \%.2f mas, de forma mais exata, é \%.11f.", piMinor, piMajor);
         
    return 0;
}
\end{lstlisting}

\subsection{Notação Científica}

Relembre o conceito de notação científica\footnote{Ler \color{links}\href{http://pt.wikipedia.org/wiki/Nota\%C3\%A7\%C3\%A3o_cient\%C3\%ADfica}{pt.wikipedia.org/wiki/Notação\_científica}}, ou seja, números no formato \(n*10^s\), ou seja, \(n\) vezes dez elevado a \(s\). Podemos utilizar notação científica nas variáveis do tipo \textit{float} e \textit{double}. Veja o seguinte exemplo:

\begin{lstlisting}
#include <stdio.h>
 
int main() {    
    float num = 24E-5; // 24 x 10 elevado a -5
    printf("\%f\n", num); // imprime: 0.000240
 
    num = 2.45E5; //2.45 x 10^5
    printf("\%.0f", num);  // imprime: 245000
 
    return 0;
}
\end{lstlisting}

\section{Caracteres - \texttt{char}}

O tipo char é um género de dados que nos permite armazenar um único carácter. É declarado da seguinte forma:

\begin{lstlisting}
char letra = 'P';
\end{lstlisting}

Como pode visualizar, a variável \texttt{letra} agora contém o carácter \quotes{P}. Pode, ao invés de utilizar este tipo de notação, utilizar números hexadecimais, octais e decimais. Em C pode-se utilizar ASCII.

\begin{defi}
\textbf{ASCII} (do inglês \textit{American Standard Code for Information Interchange}; em português "Código Padrão Americano para o Intercâmbio de Informação") é um conjunto de códigos binários que codificam 95 sinais gráficos e 33 sinais de controlo. Dentro desses sinais estão incluídos o nosso alfabeto, sinais de pontuação e sinais matemáticos. No \textbf{Anexo I} encontra uma tabela ASCII com caracteres que podem ser utilizados em C.
\end{defi}

\subsection{Função \texttt{printf} e caracteres}

Utiliza-se \textit{\%c} quando se quer imprimir o valor de uma variável dos tipo \texttt{char} dentro de uma frase. A variável deve ser colocada nos parâmetros seguintes, por ordem de ocorrência. Exemplo:

\begin{lstlisting}
#include <stdio.h>
 
int main() {    
    char letra = 'P';
 
    printf("O nome Pplware começa por %c.", letra);   
    return 0;
}
\end{lstlisting}