Neste capítulo é abordada uma parte fundamental da programação funcional, ou seja, as funções e procedimentos.

\begin{defi}
Uma \textbf{função} é um bloco de código, que tal como o próprio nome indica, tem uma função própria, ou seja, que serve para um finalidade específica.
\end{defi}

\begin{defi}
\textbf{Procedimentos} são blocos de código que contêm uma função específica. A diferença entre funções e procedimentos é que os procedimentos não retornam qualquer valor.
\end{defi}

Ao invés de colocarmos um determinado trecho de código diversas vezes, basta criarmos uma função com esse código e de cada vez que for necessário, precisamos apenas de invocar a função. As funções permitem-nos, por exemplo, reutilizar código.

\begin{defi}
\textbf{Invocar} uma função é o nome que se dá quando o nome de uma função é mencionado no código e esta é \quotes{chamada}.
\end{defi}

Durante os seus anos de programador irá criar milhares de funções para os mais diversos propósitos. Então é recomendável que vá guardando as funções que utiliza porque futuramente poderá vir a precisar delas.

Existe uma função que tem vindo a ser sempre utilizada: o \texttt{main}. É a função principal do programa; aquela que é executada automaticamente quando um programa é iniciado. 

\section{Criação de funções}

Então, o primeiro passo a dar é saber como se criam funções.

\paragraph{Sintaxe}

\begin{lstlisting}
tipo_dado_retorno nome_da_funcao(parametros) {
    // conteúdo da função
}
\end{lstlisting}

\textbf{Onde:}

\begin{itemize}
\item \textbf{tipo\_dado\_retorno} corresponde ao tipo de dados que a função vai devolver (retornar);
\item \textbf{nome\_da\_funcao} corresponde ao nome da função em questão;
\item \textbf{parametros} corresponde aos parâmetros da função, algo que será abordado mais à frente.
\end{itemize}

Imagine que precisa de um procedimento que imprime a mensagem \quotes{Hello World!} no ecrã. Poderia fazer isso da seguinte forma:

\begin{lstlisting}
#include <stdio.h>
 
int main() {
    dizHello();
    return 0;
}
 
void dizHello() {
    printf("Olá Mundo!\n");
}
\end{lstlisting}

A função chama-se \texttt{dizHello} e não retorna nenhuns dados (\texttt{void}). O que está contido entre chavetas é o código que é executado quando o procedimento é chamado.

\subsection{Argumentos e parâmetros}

No exemplo acima é possível visualizar que tanto na invocação da função, como na sua definição foi colocado um par de parênteses sem nada. Isto acontece porque não existem parâmetros.

\begin{defi}
\textbf{Parâmetros} é o conjunto de elementos que uma função pode receber (\textit{input}).
\end{defi}

Imagine agora que necessita de uma função que efetua uma soma: como enviamos os valores para uma função? Definindo o nome e o tipo de parâmetros. Ora veja:

\begin{lstlisting}
void soma(int n1, int n2) {
    int soma = n1 + n2;
}
\end{lstlisting}

A função representada no trecho acima, denominada \texttt{soma}, tem dois parâmetros, o \texttt{n1} e o \texttt{n2}, ambos do tipo \texttt{int}. Para utilizar a função, basta então proceder do seguinte modo:

\begin{lstlisting}
soma(4, 6);
\end{lstlisting}

Os números 4 e 6 (que podiam ser quaisquer outros) são denominados argumentos. A função armazena os dois argumentos na variável \texttt{soma} que apenas está disponível dentro da função, o valor da soma entre os dois números.
 
\begin{defi}
\textbf{Argumento} é o nome dado a um valor que é enviado para a função.
\end{defi}

\subsection{Retorno de uma função}

Geralmente uma função retorna um valor de forma a poder ser utilizado para outro fim. Para isso, ao invés de colocarmos \texttt{void} na declaração da função, devemos colocar qualquer outro tipo de dados.

Imagine que quer que uma função retorne o valor de uma soma entre dois números. Poderia proceder da seguinte forma:

\begin{lstlisting}
int soma(int n1, int n2) {
    return n1 + n2;
}
\end{lstlisting}

A função acima retorna dados do tipo \texttt{int} e tem dois parâmetros: o \texttt{n1} e \texttt{n2}. Podemos aplicar esta função, por exemplo, no seguinte contexto, onde o programa efetua a soma de dois números dados pelo o utilizador.

\begin{lstlisting}
#include <stdio.h>
 
int main() {
    int a, b;
 
    printf("Insira o primeiro número: ");
    scanf("%d", &a);
 
    printf("Insira o segundo número: ");
    scanf("%d", &b);
 
    printf("A soma é %d", soma(a,b));
    return 0;
}
 
int soma(int n1, int n2) {
    return n1 + n2;
}
\end{lstlisting}

\section{Algumas funções úteis}

Aqui encontram-se algumas funções que são úteis no seguimento deste livro. Algumas são relativas à entrada e saída de dados, ou seja, funções que podem receber dados inseridos pelo utilizador e funções que permitem mostrar dados no ecrã. Outras relativas à matemática. Ao longo do livro, outras funções serão abordadas.

\subsection{Função \texttt{puts}}

A função \texttt{puts} serve simplesmente para imprimir texto no ecrã. Ao contrário da função \texttt{printf} não nos permite imprimir texto formatado.

\paragraph{Sintaxe}

\begin{lstlisting}
puts("frase");
\end{lstlisting}

Ao imprimir uma frase com esta função, o carácter correspondente a uma nova linha é sempre adicionado ao final da mesma.

Imagine agora que precisa de imprimir uma mensagem que não contem nenhum valor variável e a mensagem era \quotes{Bem-vindo ao programa XYZ!}. Poderia fazê-lo da seguinte forma:

\begin{lstlisting}
puts("Bem-vindo ao programa XYZ!");
\end{lstlisting}

\subsection{Função \texttt{scanf}}

A função \texttt{scanf} permite-nos obter diversos tipos de dados do utilizador. A utilização desta função é semelhante à da já conhecida função \texttt{printf}; são como um \quotes{espelho} uma da outra. A \texttt{scanf} reconhece o \textit{input} e a \texttt{printf} formata o \textit{output}.

\paragraph{Sintaxe}

\begin{lstlisting}
scanf(fraseFormatada, &variaveis...);
\end{lstlisting}

Onde:

\begin{itemize}
\item \texttt{fraseFormatada} corresponde à formatação do que irá ser imprimido com os espaços para as variáveis;
\item \texttt{variaveis} corresponde às variáveis onde vão ser armazenados os valores obtidos por ordem de ocorrência. O nome da variável deve ser sempre precedido por um \texttt{\&};
\end{itemize}

Imagine agora que vai criar um algoritmo que peça a idade ao utilizador e a imprima logo de seguida. Poderia fazer isto da seguinte forma:

\begin{lstlisting}
#include <stdio.h>
 
int main() {   
    int idade;
 
    printf("Digite a sua idade: ");
    scanf("%d", &idade);
 
    printf("A sua idade é %d", idade);
    return 0;
}
\end{lstlisting}

Como pode verificar através da sétima linha do trecho de código anterior, o primeiro parâmetro da função \texttt{scanf} deve ser uma \textit{string} com o tipo de caracteres a ser inserido. Todos os parâmetros seguintes deverão ser o nome das variáveis às quais se quer atribuir um valor, precedidos por \texttt{\&}.

Ao se executar o trecho de código anterior, a mensagem \quotes{Digite a sua idade:} irá aparecer o ecrã e o cursor irá posicionar-se logo após essa frase aguardando que um valor seja inserido. Deve-se inserir o valor e premir a tecla \textit{enter}. Depois será imprimida uma mensagem com a idade inserida.

Relembro que pode utilizar as seguintes expressões para definir o tipo de dados a ser introduzido:

\begin{itemize}
\item \texttt{\%d} $\rightarrow$ Números inteiros (\texttt{int});
\item \texttt{\%f} $\rightarrow$ Números decimais (\texttt{float} e \texttt{double});
\item \texttt{\%c} $\rightarrow$ Caracteres (\texttt{char}).
\end{itemize}

Podem-se pedir mais do que um valor com a função \texttt{scanf}. Ora veja o seguinte exemplo:

\begin{lstlisting}
#include <stdio.h>
 
int main()
{   
    int num1, num2;
 
    printf("Digite dois números: ");
    scanf("%d %d", &num1, &num2);
 
    printf("Os números que digitou são %d e %d.", num1, num2);
 
    return 0;
}
\end{lstlisting}

Assim, quando executar o código acima, terá que escrever dois números, separados por um espaço, \textit{tab} ou \textit{enter}.

\subsection{Função \texttt{getchar}}

Existe uma forma simplicíssima de pedir caracteres ao utilizador: utilizando a função \texttt{getchar}. Basta igualar uma variável à função. Esta função é recomendável quando se quer receber um único carácter numa linha.

\paragraph{Sintaxe}

\begin{lstlisting}
variavel = getchar();
\end{lstlisting}

No seguinte exemplo é pedido para inserir a primeira letra do seu nome.

\begin{lstlisting}
#include <stdio.h>
 
int main() {   
    printf("Insira a primeira letra do seu nome: ");
    char letra = getchar();
 
    printf("A primeira letra do seu nome é %c.", letra);
    return 0;
}
\end{lstlisting}

\subsection{Limpeza do \textit{buffer}}

Quando se fala em entrada e saída de dados, deve-se ter em conta o \textit{buffer} e a sua limpeza, pois é algo extremamente importante que pode fazer a diferença entre um programa que funciona e um que não funciona.

\begin{defi}
\textit{Buffer} é o nome dado à região da memória de armazenamento física que é utilizada para armazenar temporariamente dados.
\end{defi}

Ora analise o seguinte código:

\begin{lstlisting}
#include <stdio.h>
 
int main()
{   
    char letra1, letra2;
 
    printf("Insira a primeira letra do seu nome: ");
    scanf("%c", &letra1);
 
    printf("E agora a última: ");
    scanf("%c", &letra2);
 
    printf("O seu nome começa com \"%c\" e termina com \"%c\".", letra1, letra2);
 
    return 0;
}
\end{lstlisting}

Ao olhar para o código acima, provavelmente pensará que tudo irá correr como previsto: executa-se o programa, digitam-se duas letras (\quotes{X} e \quotes{Y}, por exemplo) e depois é imprimida a mensagem \quotes{O seu nome começa com X e termina com Y.}. Infelizmente, não é isso que acontece.

Se executar o algoritmo acima, irá inserir a primeira letra, clicar na tecla \textit{enter}, mas depois o programa irá chegar ao fim dizendo que o seu nome termina com a letra \quotes{} (em branco). Por que é que isto acontece? Quando a tecla \textit{enter} é premida, o programa submete a letra inserida, mas o carácter correspondente à tecla \textit{enter}, \texttt{\textbackslash n}, também fica na memória. Assim, quando é pedido um novo carácter, o que estava na memória é automaticamente submetido.

Para que isso não aconteça, basta limpar o \textit{buffer} antes de pedir dados novamente ao utilizador. No Windows pode ser utilizada a função \texttt{fflush} e em sistemas operativos Linux a função \texttt{\_\_fpurge}. Então, ficaria assim:

\begin{lstlisting}
#include <stdio.h>
 
int main()
{   
    char letra1, letra2;
 
    printf("Insira a primeira letra do seu nome: ");
    scanf("%c",&letra1);
 	
 	// stdin corresponde à entrada teclado
    fflush(stdin);
    __fpurge(stdin);
 
    printf("E agora a última: ");
    scanf("%c",&letra2);
 
    printf("O seu nome começa com \"%c\" e termina com \"%c\".", letra1, letra2);
 
    return 0;
}
\end{lstlisting}

A utilização deste tipo de funções não é recomendável, pois não é uma convenção da linguagem C. Então pode tomar ações diferentes dependendo do compilador. O recomendável é utilizar funções que não incluam 	\quotes{lixo} quando é necessário ler algo do utilizador.

\subsection{Função \texttt{rand}}

Durante a sua jornada no mundo da programação irá precisar de gerar números aleatórios para os mais diversos fins. Em C podemos gerar números aleatórios recorrendo à função \texttt{rand}. Para poder utilizar esta função deve-se incluir a biblioteca \texttt{stdlib.h}.

\paragraph{Sintaxe}

\begin{lstlisting}
int numero = rand();
\end{lstlisting}

Imagine que precisa gerar um número aleatório entre 1 e 10. Em primeiro lugar teria que obter o resto da divisão de \texttt{rand} por 10 e somar 1. Ora veja:

\begin{lstlisting}
int numero = (rand() % 10) + 1;
\end{lstlisting}

Se experimentar executar um algoritmo que contenha a linha de código acima, irá verificar que o número gerado é sempre o mesmo, mas ninguém quer que o número gerado seja sempre o mesmo. Então, temos que \quotes{semear} a \quotes{semente} que vai dar origem ao número. Para o fazer deve-se recorrer à função \texttt{srand}.

A função \texttt{srand} permite adicionar um número como ponto de partida de forma a gerar um número aleatório. Podemos, por exemplo, gerar um número baseado na hora e tempo atuais. Ora veja um exemplo:

\begin{lstlisting}
#include <stdio.h>
#include <stdlib.h>
#include <time.h>

int main() {
	/* "semeia-se" utilizando o tempo atual num 
	tipo unsigned, ou seja, só com valores positivos */
	srand((unsigned)time(NULL));
	int numero = (rand() % 10) + 1;
	
	printf("O número gerado é %d.", numero);
	return 0;
}
\end{lstlisting}

Agora, mesmo que execute o mesmo código várias vezes, irá verificar que o número gerado é diferente na maioria das vezes. Não se esqueça de incluir a biblioteca \texttt{time.h} para poder utilizar a função \texttt{time()}.
